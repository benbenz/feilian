\documentclass[a4paper]{article}
\usepackage{graphicx}
\usepackage{twocolceurws}


\title{XPath Agent: An Efficient XPath Programming Agent Based on LLM for Web Crawler}

\author{
Li Yu \\ \\ \\ lijingyu68@gmail.com
\and
Bryce Wang \\ Stanford University  \\ brycewang2018@gmail.com
\and
Hazul \\ Research Dept.\\
                Science City, Sci 88088 \\ yvo@science.rdept.net
}

\institution{}


\begin{document}
\maketitle

\begin{abstract}
We introduce XPath Agent, a production-ready XPath programming agent tailored for web crawling tasks. A standout feature of XPath Agent is its capability to automatically program XPath queries from a set of sampled web pages. To illustrate its efficacy, we benchmark XPath Agent against a state-of-the-art XPath programming agent across a suite of web crawling tasks. Our findings reveal that XPath Agent excels in F1 score with minimal compromise on accuracy, while significantly reducing token usage and increase clock-time efficiency. The well designed 2 stage pipelines makes it readily integrable into existing web crawling workflows, thereby saving time and effort in manual XPath query development.
\end{abstract}


\section{Introduction}

Web scraping \cite{khder2021web} automates data extraction from websites, vital for modern fields like Business Intelligence. It excels in gathering structured data from unstructured sources like HTML, especially when machine-readable formats are unavailable. Web scraping provides real-time data, such as pricing from retail sites, and can offer insights into illicit activities like darknet drug markets.

The advent of HTML5 \cite{TABARES2021101529} has introduced significant complexities to automated web scraping, exacerbating issues of fragmentation and protocol diversity that have long challenged the development of web standards. These complexities stem from the enhanced capabilities and dynamic nature of HTML5, which require more sophisticated methods to accurately extract and interpret data. As traditional web scraping techniques struggle to keep pace with these advancements, there is a growing need for innovative solutions to navigate the intricacies of modern web technologies.

The development of Large Language Models (LLM) has emerged as a promising avenue. LLMs, with their advanced natural language processing capabilities, offer a new paradigm for understanding and interacting with web content. AutoWebGLM\cite{lai2024autowebglmlargelanguagemodelbased} demonstrated significant advancements in addressing the complexities of real-world web navigation tasks, particularly in simplifying HTML data representation to enhancing it's capability. By leveraging reinforcement learning and rejection sampling, AutoWebGLM enhanced its ability to comprehend webpages, execute browser operations, and efficiently decompose tasks.

Instead of fine-tuning LLMs, AutoScraper\cite{huang2024autoscraperprogressiveunderstandingweb} adopt a simplified technique which involves traversing the webpage and constructing a final XPath using generated action sequences. By focusing on the hierarchical structure of HTML and leveraging similarities across different web pages, its significantly reduces the complexity and computational overhead.

Scrolling, navigating, back and forward through the webpage to generate XPath queries is a nature way for human. But for LLM, this process is time-consuming and computationally expensive. 


\subsection{Motivation}


We assuming there are 3 core reasons why LLMs are not efficient in generating XPath queries. Firstly, LLMs are not designed to generate XPath queries. They are designed to understand and generate human-like text. Secondly, web pages are lengthy and complex, full of task unrelated information. Those information distract the LLMs from generating the correct XPath queries. Thirdly, a good XPath query should be generalizable across different web pages. However, LLMs are context limited, which means they can only generate XPath queries based on the context they have seen.

Based on the above insights, we propose a novel approach to generate XPath queries using LLMs. Our approach is designed to be efficient, accurate, and generalizable. We aim to reduce the number of steps required to generate a well-crafted XPath query, reduce the computational overhead, and improve the generalizability of the XPath queries generated by LLMs.

In order to increase the efficiency of XPath query generation, we also employed LangGraph. Which is a graph-based tool set which we can define the whole pipeline in a graph-based manner and execute it in parallel. Which significantly reduce the time required to generate XPath queries.

\subsection{Our Contributions}



\section{Related Work}
\subsection{LLMs and Direct Webpage Information Extraction}
Large Language Models (LLMs) have demonstrated remarkable abilities in replicating human-like task execution, particularly in web data extraction. Recent advancements have enabled LLMs to process entire web pages post-crawling, facilitating direct extraction of structured information such as product details and prices. This method capitalizes on the LLMs' capabilities to interpret complex HTML structures without manual rule-based configurations, enhancing adaptability across dynamic web environments (Ahluwalia et al., 2024). However, challenges remain, particularly in ensuring factual accuracy and avoiding excessive reliance on large, computationally intensive models (Xu et al., 2024). Despite these hurdles, this approach offers a promising alternative to traditional web scraping, as demonstrated by models like NeuScraper, which achieve improved accuracy by integrating neural networks for direct HTML text extraction (Xu et al., 2024).

\subsection{LLMs and Webpage Content Simplification}
An alternative approach to direct information extraction involves simplifying web content through LLMs before targeting specific data points. This method filters out redundant HTML elements, reducing the size and complexity of web data while retaining the relevant information for extraction. By employing techniques such as chunking and semantic classification, LLMs like those used in RAG models can focus on high-value segments, improving processing efficiency (Ahluwalia \& Wani, 2024) Studies have shown that simplifying web pages using LLMs significantly enhances their performance in dynamic environments, especially when dealing with varied and noisy HTML structures (Deng et al., 2023). Moreover, this approach aligns with efforts to reduce the cognitive load on models, particularly in contexts requiring rapid, scalable information extraction (Sàrl and Godin, 2023).

\subsection{LLMs and XPaths for Information Extraction}
Generating generalizable XPath queries using LLMs has emerged as a robust method for automating web information extraction across structurally similar websites. This approach leverages LLMs’ understanding of document structures to create XPath queries that dynamically adapt to minor variations in web page layouts, increasing the scalability of extraction systems. Tools like TREEX, which integrate decision tree learning, allow for the synthesis of XPaths that balance precision and recall across multiple web pages, even those with unseen structures (Omari et al., 2024). This technique significantly reduces the need for manual intervention and facilitates the creation of highly efficient and reusable extractors for tasks such as price comparison and product aggregation across e-commerce platforms (Huang et al., 2024). 
\section{Methodology}

...

\subsection{Information Extraction }

...

\subsection{XPath Program }

...

\section{Experiments }

...

\section{Conclusion }

Third level headings must be flush left, initial caps and bold.
One line space before the third level heading and $1/2$ line
space after the third level heading.

\paragraph{Fourth Level Heading}

Fourth level headings must be flush left, initial caps and roman type.
One line space before the fourth level heading and $1/2$ line
space after the fourth level heading.

\subsection{Citations In Text}

...

\subsubsection{Footnotes}

Indicate footnotes with a number\footnote{This is a sample footnote} in
the text. Place the footnotes at the bottom of the page they appear on.
Precede the footnote with a vertical rule of 2 inches (12 picas).

\subsubsection{Figures}

All artwork must be centered, neat, clean and legible. Do not use pencil
or hand-drawn artwork. Figure number and caption always appear after the
the figure. Place one line space before the figure, one line space
before the figure caption and one line space after the figure caption.
The figure caption is initial caps and each figure is numbered
consecutively.

Make sure that the figure caption does not get separated from the
figure. Leave extra white space at the bottom of the page to avoid
splitting the figure and figure caption.

Below is another figure using LaTeX commands.



\subsubsection{Tables}

All tables must be centered, neat, clean and legible. Do not use pencil
or hand-drawn tables. Table number and title always appear before the
table.

One line space before the table title, one line space after the table
title and one line space after the table. The table title must be
initial caps and each table numbered consecutively.

\begin{table}[ht]
\begin{center}
\caption{Sample Table}

\bigskip

\begin{tabular}{|l|l|r|}
\hline
A & B & 1\\ \hline
C & D & 2\\
E & F & 3\\ \hline
\end{tabular}
\end{center}
\end{table}


\subsubsection{Handling References}

Use a first level heading for the references. References follow the
acknowledgements.


\subsubsection{Acknowledgements}

Use a third level heading for the acknowledgements. All acknowledgements
go at the end of the paper.


% \addbibresource{refs.bib}
\bibliographystyle{plain} 
\bibliography{refs}


\end{document}
